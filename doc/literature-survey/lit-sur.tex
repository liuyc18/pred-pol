\documentclass{article}
\usepackage{amsmath, amssymb, hyperref}
\usepackage[a4paper, vmargin={0.3in,0.3in}]{geometry}

\title{Literature Survey on Predictive Policing}
\author{\small{Group 5: LI LINGYU, LIU YICHAO, WU JINGYAN, YANG QINGSHAN, YE FANGDA}}
\date{}

\begin{document}

\maketitle

\section{Question Definition}

Predictive Policing is defined as \textit{"the application of analytical techniques, particularly quantitative techniques, to identify likely targets for police intervention and prevent crime or solve past crimes by making statistical predictions."} It can be categorized into four main classes:
\begin{enumerate}
    \item Predicting Places of Increased Crime Risk
    \item Predicting Potential Offenders
    \item Predicting Group/Population Crime Patterns
    \item Predicting Potential Victims
\end{enumerate}

\section{Summary of Existing Systems and Their Effectiveness}

\subsection{PredPol}
PredPol, introduced around 2012, became one of the most commonly implemented predictive policing systems in the United States. It focuses on place-based or \textit{hotspot} policing, using primarily three types of historical data: crime type, location, and time. By analyzing these data, it generates risk maps highlighting areas with a high probability of crime.

Two core mathematical ideas behind PredPol are the reaction-diffusion model and the Epidemic-Type Aftershock Sequence (ETAS) model:
\begin{itemize}
    \item \textbf{Reaction-Diffusion Model}: Describes the spread of crime in a city as a chemical reaction. It assumes that crime spreads from one location to another similar to how chemicals diffuse in a liquid.
    \item \textbf{ETAS Model}: Treats the dynamic occurrence of crime as a continuous-time, discrete-space epidemic-type aftershock sequence point process. The model assumes that crimes are not isolated events but are influenced by past crimes in the same area.
\end{itemize}

\textbf{Effectiveness}: Early assessments in cities like Santa Cruz, California, documented an approximately 19\% decrease in burglaries within designated zones. In Los Angeles, pilot programs reported notable reductions in property crime, indicating that the model can help law enforcement allocate resources efficiently.

\textbf{Drawbacks}: Studies in 2019, such as those conducted in Oakland, revealed that reliance on historical crime data can reinforce existing disparities, leading to heightened police scrutiny of marginalized communities, raising ethical concerns regarding bias and over-policing.

\subsection{Other Systems and Approaches}
\begin{itemize}
    \item \textbf{HunchLab}: Integrates crime data, geographic and temporal data (e.g., weather, school schedules), and demographic information to forecast crime likelihood.
    \item \textbf{Crime Anticipation System (CAS) (Netherlands)}: Uses geographic clustering and temporal crime trends to predict future crimes based on recent crimes in nearby areas.
    \item \textbf{Risk Terrain Modeling (RTM)}: Incorporates environmental factors like abandoned buildings and liquor stores, using machine learning to analyze sociological data.
    \item \textbf{People-Based or Group-Based Policing}: Attempts to identify individuals or groups at high risk of committing or being victims of crime, particularly gun violence, though ethical concerns persist.
\end{itemize}

\section{Review of the Modelling Approaches}

\subsection{Reaction-Diffusion Model}
In a discrete-time model, incorporating the \textit{broken window effect}, we have:
\begin{equation}
    B_s(t + \delta t) = \left[ B_s(t) + \frac{\eta \ell^2}{z} \Delta B_s(t) \right] (1 - \omega \delta t) + \theta E_s(t)
\end{equation}
where $B_s(t)$ models the risk of a site being attacked at time $t$. Other parameters include:
\begin{itemize}
    \item $\omega$: Time scale over which repeat victimizations are most likely.
    \item $\theta$: Multiplier of $E_s(t)$, the number of burglary events at site $s$ since time $t$.
    \item $\eta$: Measures the significance of neighborhood effects ($0 \leq \eta \leq 1$).
    \item $\ell$: Grid size, $z$: Number of neighboring sites.
    \item $\Delta$ (Discrete Laplacian):
    \begin{equation}
        \Delta B_s(t) = \frac{1}{\ell^2} \left( \sum_{s' \sim s} B_{s'}(t) - z B_s(t) \right)
    \end{equation}
\end{itemize}

By taking limits as $\ell$ and $\delta t \to 0$, we obtain the differential equation:
\begin{equation}
    \frac{\partial B}{\partial t} = \frac{\eta D}{z} \nabla^2 B - \omega B + \epsilon D \rho A.
\end{equation}

\subsection{Using the EM Algorithm in the ETAS Model}
The ETAS model assumes a continuous-time, discrete-space problem where each event generates $N \sim \text{Poisson}(\theta)$ direct offspring events. The event rate in box $n$ at time $t$ is given by:
\begin{equation}
    \lambda_n(t) = \mu_n + \sum_{t_n^i < t} \theta \omega e^{-\omega (t - t_n^i)}.
\end{equation}

The Expectation-Maximization (EM) algorithm is used for parameter estimation. In the E-step:
\begin{equation}
    p_n^{ij} = \frac{\theta \omega e^{-\omega (t_n^i - t_n^j)}}{\lambda_n(t_n^j)}, \quad p_n^j = \frac{\mu_n}{\lambda_n(t_n^j)}.
\end{equation}

The M-step updates parameters:
\begin{equation}
    \omega = \frac{\sum_n \sum_{i < j} p_n^{ij}}{\sum_n \sum_{i < j} p_n^{ij} (t_n^j - t_n^i)}.
\end{equation}

More details can be found \href{https://github.com/arun-ramamurthy/pred-pol/blob/master/doc/Rederivation\%20of\%20Mohler\%20et\%20al.pdf}{here}.

\section{Feature Engineering Techniques}
\begin{itemize}
    \item Temporal Features: Day-of-week, time-of-day, seasonal flags.
    \item Spatial Features: Clusters, geohashes, distance to police stations.
    \item Crime Type Encoding: One-hot encoding, risk classification.
    \item Aggregated Rates: Crime/arrest counts per area/time window.
\end{itemize}

\section{Evaluation Metrics and Their Appropriateness}
\begin{itemize}
    \item Prediction Success Rate
    \item Changes in Crime Rate
    \item False Positive/Negative Rates
\end{itemize}

However, these metrics face challenges related to bias, lack of transparency, and oversimplification of social complexities.

\end{document}
