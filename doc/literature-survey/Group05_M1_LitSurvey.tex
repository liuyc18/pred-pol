\documentclass[11pt]{article}
\usepackage{fontspec}
% \setmainfont{Times New Roman} 
\usepackage{amsmath, amssymb, hyperref}
\usepackage[a4paper, vmargin={0.45in,0.45in},hmargin={1in,1in}]{geometry}
\usepackage{enumitem}
\usepackage{titlesec}
\usepackage{setspace}
\usepackage{nopageno}
\spaceskip=0pt
\setlength{\parskip}{0pt}
\titlespacing*{\section}{0pt}{0.4\baselineskip}{0.4\baselineskip}
\titlespacing*{\subsection}{0pt}{0.3\baselineskip}{0.3\baselineskip}
\titlespacing*{\paragraph}{0pt}{0.16\baselineskip}{0.5\baselineskip}


\title{\raggedright Literature Survey on Predictive Policing}
\author{\raggedright \small{LI LINGYU, LIU YICHAO, WU JINGYAN, YANG QINGSHAN, YE FANGDA}\hfill \small{Group 5}}
\date{} 

\begin{document}

% \maketitle
\makeatletter
\begin{flushleft}
    {\LARGE \@title} \\[-0.5em] % Title size and spacing
    \rule{\textwidth}{1pt} \\[0em] % Horizontal line
    {\large \@author} \\[0em] % Author size and spacing
    % {\small \@date} \\[2em] % Date size and spacing
\end{flushleft}
\makeatother
\section{Problem Definition}
Predictive Policing is defined as \textit{"the application of analytical techniques, particularly quantitative techniques, to identify likely targets for police intervention and prevent crime or solve past crimes by making statistical predictions."} \cite{RAND2013}. Considering the research subjects, there are four main categories: predicting \textit{Places of Increased Crime Risk}, predicting \textit{Potential Offenders}, predicting \textit{Group/Population Crime Patterns} and predicting \textit{Potential Victims}. 


\section{Existing Systems and Modelling Approaches}
Depending on different perspecitves, the prediction can be either \textbf{place-based} or \textbf{people-based}, or a combination of both. Place-based prediction focuses on the relationship between geographic locations and crime rates, which is effective but proved to be biased against racial minorities and would possibly lead to negative loops\cite{brantingham2018biased}. People-based prediction, on the other hand, aims to identify individuals or groups at high risk of committing or being victims of crime, which is also controversial due to ethical concerns, such as privacy and discrimination. Currently many of  PredPol\cite{ferguson2017rise,ensign2017runaway}, Chicago's Strategic Subject List (SSL)\cite{tayebi2016social}, Risk Terrain Modeling (RTM)\cite{caplan2011risk}, Real-Time Crime Centers (RTCC)\cite{ratcliffe2011philadelphia}, HunchLab\cite{hunchlab_ratcliffe_philadelphia_2019}, etc. are widely used in the U.S. and other countries.

\subsection{Place-Based: PredPol (Geolitica)}
\paragraph{Modelling Methods} PredPol\cite{ferguson2017rise,ensign2017runaway}, introduced around 2012, is one of the most commonly implemented predictive policing systems in the U.S.\cite{pechini1967washington} and thus usually singled out in discussions. It learns from physical phenomena like chemical reactions as well as earthquakes and aftershocks to predict \textbf{place-based} or \textbf{hotspot} crime. Specifically, PredPol perceives crimes as contagions that spread from one location to another, similar to earthquakes and aftershocks, while adding "hotspot policing" could inhibit the process like in a chemical reaction. 
Two core mathematical ideas behind these are \textit{the reaction-diffusion model}\cite{react-diff-short2008statistical,react-diff-short2010nonlinear} and \textit{the Epidemic-Type Aftershock Sequence (ETAS) model}\cite{ETAS_ogata1988statistical}. More details are listed below.
% in the \hyperref[sec:appendix]{Appendix}.

% \section*{Appendix: Mathematical Details in PredPol Modelling Approaches}
% \label{sec:appendix}
\paragraph{Reaction-Diffusion Model\cite{react-diff-short2008statistical,react-diff-short2010nonlinear}} 
It assumes that crime spreads from one location to another similar to how chemicals diffuse in a liquid. We can model the risk of a site $s$ being attacked at time $t$ as:
\vspace*{-1em}
\begin{equation}
    B_s(t + \delta t) = \left[ B_s(t) + \frac{\eta \ell^2}{z} \Delta B_s(t) \right] (1 - \omega \delta t) + \theta E_s(t)
    \vspace*{-0.3em}
\end{equation}
and discret Laplacian $\Delta B_s(t)$ means the the crime would spread to its neighbors, also known as \textit{broken window effect}:
\vspace*{-1em}
\begin{equation}
    \Delta B_s(t) = \frac{1}{\ell^2} \left( \sum_{s' \sim s} B_{s'}(t) - z B_s(t) \right)
    \vspace*{-0.3em}
\end{equation}
Here $\omega$ is the time scale over which repeat victimizations are most likely. $\theta$ is the multiplier of $E_s(t)$, which is the number of burglary events at site $s$ since time $t$. $\ell$, $z$ and $\eta$ are grid size, number of neighboring sites and the significance of neighborhood effects ($0 \leq \eta \leq 1$), respectively. We can achieve a continuous setting by taking limits as $\ell$ and $\delta t \to 0$.

\paragraph{ETAS Model\cite{ETAS_ogata1988statistical}}
ETAS Model treats the dynamic occurrence of crime as a continuous-time, discrete-space epidemic-type aftershock sequence point process. The model assumes that crimes are not isolated events but are influenced by past crimes in the same area, so each event generates $N \sim Poisson(\theta)$ direct offspring events. The event rate in box $n$ at time $t$ is given by:
\vspace*{-0.5em}
\begin{equation}
    \lambda_n(t) = \mu_n + \sum_{t_n^i < t} \theta \omega e^{-\omega (t - t_n^i)}
    \vspace*{-0.8em}
\end{equation}
where $\mu_n$ is the background rate of crime in box $n$, while $t_n^i$ are the times of events in box $n$ in the history of process. $\omega$ is the decay rate of influence of past events. The Expectation-Maximization (EM) algorithm can be used for parameter estimation\cite{lum_etas_em}. 

% To be general, the expectation, or E-step, sets
% \begin{equation}
% p_n^{ij} = \frac{\theta \omega e^{-\omega (t_n^i - t_n^j)}}{\lambda_n(t_n^j)}, \quad p_n^j = \frac{\mu_n}{\lambda_n(t_n^j)},
% \end{equation}
% where $\theta \omega e^{-\omega t}$ is called \textit{the triggering kernel} that models "near-repeat" or "contagion" effects in crime data. The maximization, or M-step, sets
% \begin{equation}
% \omega = \frac{\sum_n \sum_{i < j} p_n^{ij}}{\sum_n \sum_{i < j} p_n^{ij} (t_n^j - t_n^i)},
% \end{equation}

% \begin{equation}
% \theta = \frac{\sum_n \sum_{i < j} p_n^{ij}}{\sum_n \sum_j 1},
% \end{equation}

% \begin{equation}
% \mu = \frac{\sum_n \sum_j p_n^j}{T},
% \end{equation}
% where $T$ is the length of the time window of observation. The algorithm iterates between the E-step and M-step until convergence.
\paragraph{Effectiveness} 
Early assessments in Santa Cruz saw a 19\% decrease in burglaries, while Los Angeles pilot programs reported notable reductions in property crime, demonstrating the model’s effectiveness in resource allocation for law enforcement. However, a 2019 study in Oakland revealed that relying on historical crime data can reinforce disparities, leading to heightened police scrutiny of marginalized communities and raising ethical concerns about bias and over-policing.

\subsection{People-Based: Chicago's Strategic Subject List (SSL)}
\paragraph{Modelling Methods} The key idea behind the SSL is to constrcut a \textbf{co-offending network} model of crime\cite{tayebi2016social,burcher2020social}, which is a graph representation of the relationships between individuals who have committed crimes together. Upon the network predictions and recommendations are made to the police department.

\paragraph{Effectiveness} The SSL has been used in Chicago since 2013 to identify individuals at high risk of being involved in gun violence. The model has been criticized for its lack of transparency as well as for the ethical implications of targeting individuals based on their social networks. 

\subsection{Hybrid: Risk Terrain Modeling (RTM) and Real-Time Crime Centers}
\paragraph{Modelling Methods} RTM\cite{caplan2011risk} combines environmental factors like abandoned buildings and liquor stores with sociological data to reduce potential biases. RTCCs\cite{ratcliffe2011philadelphia} use real-time data to identify crime patterns and allocate resources accordingly. Both of them are hybrid systems that combine place-based and people-based approaches.

\paragraph{Effectiveness} RTM has been used in cities like Philadelphia to predict crime risk and allocate resources accordingly. RTCCs have been implemented in cities like New York and Los Angeles to monitor crime patterns in real time and respond to incidents more effectively. Both systems have been criticized for their potential to reinforce existing biases and for the ethical implications of targeting individuals based on their social networks.

\subsection{Review of Mathematical Ideas in Modelling Approaches}
\paragraph{Differential Equations} Used in a continuous setting to model spatially and temporally.
\paragraph{Graphical Model Inference} Used to model social networks and co-offending relationships, with methods like Sum-product, MCMC and Variational Inference.
\paragraph{Optimization Techniques} Methods like SGD or the Expectation-Maximization (EM) algorithm are used to estimate parameters in models like the ETAS.
\paragraph{Machine Learning} Techniques like Random Forests, Gradient Boosting, Clustering and Neural Networks are used to predict crime risk based on historical data.

\section{Feature Engineering Techniques}
\paragraph{Temporal Features} Includes day-of-week, time-of-day, and seasonality flags such as holidays.
\paragraph{Spatial Features} Incorporates clusters, proximity measures, geohashes, distance to police stations, and environmental factors like bars and crime hotspots.
\paragraph{Crime Type Encoding} Utilizes one-hot encoding, classification by severity, and further categorization into high- and low-risk groups for arrests.
\paragraph{Aggregated Rates} Considers crime counts per area and time window, arrest counts per area and time window, and flags for repeat domestic incidents.

\section{Evaluation Metrics and Their Appropriateness}
\paragraph{Prediction Success Rate} Measures the proportion of actual crimes that occur within the areas designated as crime hot spots. For example, a 2015 study by PredPol claimed that its model accurately predicted 4.7\% of crimes and was twice as accurate as traditional methods.
\paragraph{Changes in Crime Rate} Based on historical records, Santa Cruz Police Department observed an 11\% reduction in residential burglaries and a 27\% in robberies after deploying PredPol.
\paragraph{False Positive/Negative Rates} Simulation studies suggest that feedback loops may cause the system to over-predict crime in certain neighborhoods. For example, an analysis of drug-related crime forecasts in Oakland found that communities of color were misidentified at roughly twice the rate of predominantly white areas.

\section{Challenges and Ethical Considerations}
Limited transparency in predictive policing raises concerns about inequities, especially among marginalized groups already facing disproportionate surveillance. By prioritizing spatial and temporal factors, these tools may oversimplify social complexities, overlooking deeper crime drivers like poverty and education inequality. While short-term crime reductions may occur, the long-term consequences, such as eroding community trust and increasing tensions with law enforcement, remain significant challenges.
\newpage
\bibliographystyle{plain}
\bibliography{refs}

\end{document}
